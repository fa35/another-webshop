\newcommand{\VORNAME}{Nicolai}
\newcommand{\NACHNAME}{Jablonski}
\newcommand{\KLASSE}{FA35}
\documentclass[a4paper,oneside,12pt]{scrartcl}

\usepackage{geometry}
\geometry{a4paper,left=20mm,right=20mm, top=18mm, bottom=18mm} 

\usepackage[utf8]{inputenc}
\usepackage[T1]{fontenc}
\usepackage{ngerman,url}
\setcounter{secnumdepth}{-1}
\sloppy

\usepackage{scrpage2}
\setlength{\headheight}{1.5\headheight}
\pagestyle{scrplain}
\clearscrplain
\lohead[\footnotesize\KLASSE, \VORNAME{} \NACHNAME]{}
\cohead[\Large{\textsc{AS Projekt}}\\\scriptsize Dokumentation Versand - Version 1.1]{}
\rohead[\footnotesize\today]{}

\begin{document}

\section{Vorbedinungen}
Es soll ein Webshop zum Bestellen von Produkten erstellt werden. Dieser
soll eine Administrationsschnittstelle beinhalten, so dass Benutzer, Mehrwertsteuersätze,
Artikelgruppen und Artikel anlegen und verwalten kann.
Der Benutzer soll sich über eine Login-Maske anmelden können.
Es wurden Beispielhafte Datenbank-Strukturen vorgegeben, die implementiert werden können.

\section{Planung}

\subsection{Benutzergruppen}
Die Benutzer des Webshops werden in 2 Benutzergruppen eingeteilt: Administratoren und Kunden.
Die Benutzergruppe wird als boolscher Wert an den Datensatz des Benutzers angehängt.

\subsection{Artikel und Artikelgruppen}
Artikel werden in Artikelgruppen aufgeteilt um die Navigation für den Benutzer zu erleichtern.
Jeder Artikel hat eine Bezeichnung, eine Beschreibung, einen Nettopreis sowie einen 
zugeordneten Mehrwertsteuersatz und eine Artikelgruppe. Der Bruttopreis wird automatisch 
berechnet und dem Benutzer angezeigt.

\subsection{Administration}
In der Administration können Benutzer, Mehrwertsteuersätze, Artikelgruppen, Zahlungsarten und Artikel anlegt
und verwaltet werden. Dazu wird es notwendig sein einen von den Kunden abgetrennten
Bereich zu erstellen. Es ist vorgesehen, dass die Administratoren sich auch um die
Bestellungen kümmern und die Bestellungen aller Benutzer sehen können.
Die Administrationsschnittstelle kann über die Hauptschnittstelle aufgerufen werden.

\subsection{Registrierung und Anmeldung}
Benutzer können entweder über einen Benutzernamen oder über seine Email-Adresse eindeutig
identifiziert werden. Email-Adressen sind allerdings immer eindeutig, da es nicht gängig ist,
dass mehrere Benutzer sich eine Email-Adresse teilen. Zur Sicherheit sollte das Password des
Benutzers nicht im Klartext gespeichert werden, sondern verschlüsselt beziehungsweise als
Hash. Die Anmeldung erfolgt für beide Benutzergruppen über die selbe Login-Maske. 

\subsection{Warenkorb}
Der Warenkorb ist die virtuelle Version eines Einkaufenswagens, in dem Artikel eingefügt und
wieder entfernt werden können. Wenn Artikel dem Warenkorb hinzugefügt werden,
so soll auch angezeigt werden, was sich in dem Warenkorb befindet.
Die Anzeige der Warenkorbes soll tabellarisch die Artikel, zu zahlende Mehrwertsteuer (aufgeschlüsselt nach Mehrwertsteuersatz) 
und den Endpreis anzeigen.
Bei dem Artikeln wird die Bezeichnung, die Menge, zugeordneter Mehrwertsteuersatz, der Einzelpreis (brutto) und
der Gesamtpreis anzeigt. Bei dem Login bleibt der Warenkorb erhalten, beim Logout wird dieser geleert.

\subsection{Bestellvorgang}
Der Bestellvorgang kann nur von angemeldeten Benutzern gestartet werden.
Der Aufruf des Bestellvorgangs erfolgt über den Warenkorb.\\
\\
Der Bestellvorgang besteht aus 2 Teilen:
\begin{itemize}
  \item Eingabe der Lieferadresse und
  \item Anzeige der Produkte im Warenkorb sowie der Auswahl der Zahlungsart und dem Bestellbutton
 \end{itemize}
Die Bestellvorgang ist erfolgreich, wenn die Lieferadresse eingegeben wurde und die
Bestellung in der Datenbank gespeichert wurde.

\subsection{Bestellungsansicht}
Die Bestellungsansicht zeigt die chronologisch umgekehrt sortierten Bestellungen (jüngste oben) analog zum Bestellvorgang an.
Dazu wird auch noch der Zeitpunkt und der Zahlungsstatus der Bestellungen anzeigt.\\
\\
Bei Kunden werden die eigenen Bestellungen anzeigt.\\
Bei Administratoren werden die Bestellungen aller Benutzer angezeigt, aufgeschlüsselt nach Benutzern. Zusätzlich kann von den Administratoren der 
Zahlungsstatus der Bestellung gesetzt werden.

\section{Umsetzung}
Das Projekt wurde in Skriptsprachen PHP und Javascript erstellt.
Zur Anzeige des Projekts wird ein Webbrowser benötigt, dabei ist empfehlungsweise
Google Chrome, Chromium oder Mozilla Firefox in der neusten Version zu nutzen.\\
Der Datenbank-Zugriff erfolgt über die in PHP integrierte Programmbibliothek mysqli.
Aus den Datenbank-Spalten werden in php Objekte erzeugt. Die Klassen zu den Objekten befinden sich unter classes/model.\\
\\
Zum Betrieb des Projekts auf dem Server wird folgende Programme in der angegeben Version oder neuer benötigt:
\begin{itemize}
 \item Apache 2.4
 \item PHP 5.6
 \item MariaDB-Server 10.0 oder MySQL-Server 5.6
\end{itemize}
Das Projekt muss entweder im DocumentRoot oder in einem Alias-Ordner liegen. Es wird HTML5 verwendet.\\
\\
Das Layout des Projektes wurde größtenteils mit div-Tags und CSS-Eigenschaften erstellt und auf die Anzeige
in Google Chrome und Chromium hin optimiert. Bei anderen Browsern kann es zu kleinen Unschönheiten im Layout kommen.\\
\\
Der Login und der Warenkorb werden in der \_SESSION-Variabel von PHP gespeichert. Dazu wird von php beim ersten Zugriff auf die Seite automatisch eine
Session gestartet, die SESSIONID, die auf die Session in PHP referenziert, erstellt und mit einem Cookie im Browser des Nutzers gespeichert. 
Die Session ist je nach Konfiguration von PHP bis der Browser geschlossen wird gültig.
Bei dem Logout wird die Session geschlossen und eine neue Session automatisch gestartet.\\
\\
Es wird die Javascript-Library jquery in der Version 2.1.3 zum Hinzufügen von Produkten in den Warenkorb genutzt,
da die Seite sonst bei dem Hinzufügen von Produkten neu geladen würde. So wird über jquery und AJAX dass
Hinzufügen von Artikeln in den Warenkorb im Hintergrund ausgeführt und der Status des Warenkorbes ohne ein 
Neuladen der Seite aktualisiert.\\
\\
Bei dem Artikel kann ein png- oder jpg-Bild des Artikels beim Erstellen oder Bearbeiten eines Artikels hochgeladen werden. 
Das Bild wird als png-Datei unter pictures mit dem SHA256-Hash der Original-Bilddatei als Dateinamen gespeichert 
und ein Thumbnail wird dabei automatisch erstellt (thumb\_[SHA256-Hash der Original-Bilddatei].png).
Bei dem Anklicken des Thumbnails wird dieses in der voller Größe angezeigt.

\subsection{Projektaufbau}
Das Projekt ist in mehrere Teile aufgebaut:
\begin{itemize}
 \item admin: enthält die Administration. Jede Administrationsaufgabe hat ihren Unterordner.
 z.B. die Artikelverwaltung im Unterordner \textquotedblleft article\textquotedblright.
 Jeder Unterordner besitzt im Normalfall 4 php-Dateien: Eine zum Erstellen (add), zum Editieren (edit), zum Löschen (delete)
 und zur Anzeige der Elemente (administration).
 \item classes: enthält Datenbankzugriffs-, Konfigurations und Steuerklassen
 \item classes/model: enthält Klassen welche die Daten aus der Datenbank halten
 \item order: enthält php-Dateien zum Warenkorb und Bestellvorgang
 \item user: enthält php-Dateien zum Login und Registrierung von Benutzern
 \item pictures: enthält Bilder, das Hintergrundbild sowie alle Bilder, die der Nutzer zu den Artikeln hochgeladen hat.
\end{itemize}

\section{Umsetzungsalternativen}
Es hat sich bei der Entwicklung gezeigt, dass auch die sogenannte WebAPI statt jquery über Javascript genutzt werden kann.
Die WebAPI ist allerdings noch experimentell und wurde erst in Mozilla Firefox 35 und Google Chrome / Chromium 40 implementiert.
Alle anderen Browser unterstützen die WebAPI nicht.

\section{Erweiterbarkeit}
Das Projekt kann um eine Benutzergruppen- und Rechteverwaltung erweitert werden, um differenzierter 
Rechte verteilen zu können, z.B. das nur eine Gruppe \textquotedblleft Lagerarbeiter\textquotedblright\  die
Bestellungen aller Kunden einsehen können oder nur Kunden Bestellungen aufgeben können.\\
\\
Die Registrierung der Benutzer kann außerdem so erweitert werden, dass der Benutzer sein Benutzerkonto
erst über eine Aktivierungsmail bestätigen muss. Zur Sicherheit kann auch ein Captcha oder ähnliches Verfahren
genutzt werden um die automatische Registrierung über Bots zu erschweren.
\end{document}